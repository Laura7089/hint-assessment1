\section{Ethernet and Time-Sensitive Networks}
% ~3.5 pages

\subsection{Ethernet}
% Part A: <1 page

% - if a frame is damaged, detected by the CRC not matching, then ethernet provides no method of guaranteed redelivery; this must be implemented somewhere higher in the OSI stack
%
% create (or steal) diagram depicting typical ethernet architecture - perhaps show full-duplex-compatible vs non?

Ethernet is a set of systems standardising 2-way device-to-device digital data exchange and communication over physical cabling.
Devices within an Ethernet network are referred to as \keyword{stations}; stations exchange \keyword{frames}: variable-length, self-contained, addressed sequences of bits which are typically single-station-to-single-station (though ``multicast'' modes of operation are available)~\cite{IEEEStandardEthernet2022,spurgeonEthernetDefinitiveGuide2000}.

The basic Ethernet standard (see \cite{IEEEStandardEthernet2022}) does not provide any mechanism to guarantee delivery of frames, nor the time taken to do so.
It provides a basic system for receiving stations to detect errors during frame transmission to a reasonable accuracy using a checksum algorithm attached to each frame\footnote{ARINC 629 likewise provides a checksum~\cite{dheerajpuniaARINC629Digital2021}, though the guarantee of integrity is weaker owing to the simplicity of the algorithm and the length of the single parity bit.}, but no method for recovery from this error --- the frame is simply dropped~\cite{odomCCNA200301Official2019}.
The protocol does not provide any way for a frame's transmission to be interrupted.

Ethernet offers two methods of operation pertinent to resource management:

\begin{description}
    \item{\keyword{Half-Duplex}}\label{def:halfduplex} offers links between devices which are only available to transmit on by a single station at a time.
        This is managed through a mechanism known as the \keyword{Media Access Control Protocol}, which is embedded in each station's Ethernet controller.
        Under this system, stations must listen to the link to determine if it is in use, then attempt to transmit their next frame; if two stations try to transmit a frame at the same time, a \keyword{collision} occurs and both devices must ``back off'' and attempt to re-transmit their frame in the same manner a random time offset in the future.
    \item{\keyword{Full-Duplex}}\label{def:fullduplex} offers links between devices which may be both transmitted and received on at the same time~\cite{spurgeonEthernetDefinitiveGuide2000}.
        This mode can only be enabled on point-to-point links which have exactly two stations with all hardware being compatible, including cabling.
        Under this system, the link is split into TX and RX channels which may be utilised simultaneously, eliminating the need to manage collisions and the MAC Protocol.
        Higher throughput and more predictable timing may be conferred by this, but this is not guaranteed; other sources of nondeterministic time still exist within the system.
        Furthermore, for networks with many stations, switches (see below) or some other form of routing are required.
\end{description}

In systems with many stations which all need to transmit different data to each other, it may be desirable to add in \keyword{Switches}\footnote{Sometimes referred to as \keyword{Switching Hubs}~\cite{spurgeonEthernetDefinitiveGuide2000}.}.
These are stations with many connection ports that function as address-aware relays: they automatically segment the network by station residency and only forward frames to segment(s) connected to the addressee\cite{spurgeonEthernetDefinitiveGuide2000}.
A switch may allow more efficient link utilisation, and fewer collisions to occur in Half-Duplex operation.
They can also allow economic hybridisation of hardware in networks, with higher-throughput links between switches and ``downgraded'' links between switches and non-switch stations as required.

\subsection{Time-Sensitive Networks: Application and Research}
% Part B: ~2 pages

``Time-Sensitive Networks'' as we know them originate from a project of the \keyword{AVB} (Audio-Visual-Bridging) task group founded in 2005 at the IEEE.
In 2012, the potential for use in industrial automation was recognised and the project was renamed~\cite{imtiazProposalIntegrateProcess2011,zezulkaTimeSensitiveNetworkingCommunication2019}.
TSN has since seen varying adoption across the Automotive, Aviation and Production industries.

In the Automotive industry, dedicated data bus systems designed for automotive use like CAN~\cite{RoadVehiclesController2024} rule supreme in inter-system communication, even being legally mandated in several major territories --- including the European Union~\cite{RegulationECNo2007}.
CAN has relatively low data throughput and typical modern implementations utilise a complex domain-based architecture with highly distributed computing resources~\cite{ashjaeiTimeSensitiveNetworkingAutomotive2021a}.
Automotive TSN proposals and research currently promote the concept of a system with a smaller set of controllers networked together over a single high-throughput (10+Gbps) ethernet network~\cite{ashjaeiTimeSensitiveNetworkingAutomotive2021a,zinnerAutomotiveArchitectureEvolution2019}, though it is unclear whether this system has been deployed in earnest.

In Aviation, the IEEE are working on the P802.1 standard~\cite{P8021DPTSN2024} for wide adoption.
The ARINC664P7 specification~\cite{ARINC664P71AIRCRAFTDATA2009} is a comparable existing standard using ethernet, with \keyword{AFDX} as the de-facto implementation~\cite{moreauxDataTransmissionSystem2005}.
AFDX is installed on many operational platforms today, including the Boeing 787 Dreamliner, the Airbus A350 and the A380 for which it was originally developed~\cite{AFDXTechnologyImprove2005,AirbusRockwellCollins13,itierA380IntegratedModular}.
It has since influenced and informed the development of TSN~\cite{pasquierAvionicsFullDuplex2015}.

In production industries, TSN has seen advancements in deployment on factory floors and production lines as a subcategory of the accepted ``Industrial Ethernet'' group of technologies.
This category is unique among the other discussed areas in that ethernet technologies with temporal budgeting and real-time extensions were already employed prior to the advent of TSN as it is now.
EtherCAT, ProfiNet and other ethernet-based systems saw wide use in the 2010s and later~\cite{linLookIndustrialEthernet2018}, largely superceding the array of FieldBus technologies that came before them.
Today, FieldBus and Ethernet-based systems are standardised under the same banner, IEC 61158~\cite{IEC611581Industrial2023}.
TSN has not yet seen inclusion in this standard, but it is expected that the standardisation of real-time-ethernet functionality will benefit the industry and allow interoperability between existing technologies~\cite{linLookIndustrialEthernet2018,sommerEthernetSurveyIts2010}.

From an academic standpoint, TSN is undergoing several different directions of research:

\begin{itemize}
    \item Network calculus has been employed in several studies to prove worst-case latency and buffer size bounds across different network configurations~\cite{maileNetworkCalculusResults2020}.
    In turn this provides theory-based evidence for the behaviour of different schedulers and network architectures.
    \item Attempts to enable TSN over wireless networks are ongoing, aiming to bring the advantages of moveable and rapidly-reconfigurable stations to bear.
    Some obstacles to a workable proof-of-concept still persist, notably time synchronisation (ie. PTP~\cite{IEEEStandardLocal2020}), but are expected to be surmountable~\cite{mildnerTimeSensitiveNetworking2019}.
    \item TSN at present cannot model dynamic traffic priority classes that may be required when a drastic change occurs in the environment; it can only operate in an environment with statically-known traffic profiles~\cite{seolTimelySurveyTimeSensitive2021}.
    \item Routing algorithms in the accepted standard leave something to be desired in redundancy and load balancing.
    Alternative systems have been proposed~\cite{atallahRoutingSchedulingTimeTriggered2020}.
\end{itemize}

\subsection{Mixed Criticality System with TSNs}
% Part C: ~1.5 pages

We use the given example subsystems to illustrate our example of a TSN for an aviation-relevant mixed criticality system.

The \keyword{Command System} is inserted, and defined as the systems that interface with the crew (primarily pilots) of the aircraft and allow them to issue commands to and view the status of other systems; in addition to encompassing ``auto-pilot'' capabilities.

We further make the following assumptions:

\begin{itemize}
    \item All subsystems must interface with the \keyword{Command System} to allow interaction from the crew
    \item \keyword{Flight Control System} encompasses the control surfaces and the actuation systems related to them
    \item \keyword{Power Distribution System} refers to routing of electrical power across the aircraft, and that all other subsystems require power delivery and thus communication with the PDS
    \item \keyword{Engine Control System} encompasses active engine controls such as throttling, shutdown etc. in addition to into-engine fuel pumping
    \item \keyword{Fuel Management System} encompasses fuel tank status data and control for inter-tank fuel transfer and balancing, and must maintain a dedicated connection with the Engine Control System to exchange information about fuel demands and consumption rates
    \item All sub-systems must interface with the \keyword{Power Distribution System} in order to negotiate their power needs
    \item All sub-systems' connections with the Command System and the Power Distribution System inherit the DAL of that subsystem; we consider failures in communication to constitute failure of the system as either or both of loss of control and loss of power will occur
    \item The \keyword{Navigation System} must transmit data to the Command System with a deterministic rate and latency
\end{itemize}

With this information, the collective communication network on the aircraft is segmented into VLANs with appropriate priorities set as defined in IEEE 802.1Q~\cite{IEEEStandardLocal2022} chosen based upon the DAL of the components contained within, shown here in figure~\ref{fig:vlans}.
Time-aware scheduler IEEE 802.1Qbv~\cite{IEEEStandardLocal2018a,ISOIECIEEE2018} is enabled on all connections and devices.
To facilitate the correct function of Time-Sensitive Scheduling, at least one independent dedicated high-priority PTP grandmaster unit (\keyword{PTP-GM}), compliant with the specific PTP subset IEE 802.1AS~\cite{IEEEStandardLocal2020} must exist on the network, with potential for one or more redundant additional units.
All stations must be able to fully comply with the 802.1Q~\cite{IEEEStandardLocal2018a}, 802.1Qbv~\cite{IEEEStandardLocal2016a} and 802.1AS~\cite{IEEEStandardLocal2020} standards to uphold the guarantees required of the system.

Figure~\ref{fig:topology} shows a reference topology for only the priority A VLAN between the Command, Flight Control, Power Distribution and Engine Control subsystems.
For visual simplicity, redundant network links (\keyword{INT-RED}) and redundant grandmaster units (\keyword{PTP-GM-RED}), among others, are not shown; however, their presence should be expected --- it will be assumed that the shown network is duplicated wholly and independently once (\keyword{ALL-RED}), including Ethernet interfaces.
Such a redundancy would additionally require switching logic (\keyword{INT-SWC}) on peripherals to avoid duplication of packets.
Fault detection on all network(s) should be able to detect malformed frames (\keyword{INT-FD}).

In order to manage temporal resources closely and effectively, the system might configure the 803.1Qbv scheduler to allocate time within network cycles based on the priority of the VLANs within~\cite{IEEEStandardLocal2016a}.
Once required transmission/reception rates and latencies of subsystems are known (these can be calculated from their expected performance envelopes), this can be combined with additional time extensions based on their DAL class to ensure their respective VLANs are allocated the time they require for correct and safe operation.
Based on the large throughput boost to be expected from even the most modest Ethernet implementations over ARINC 629, the proposed system should have plenty of temporal resources available for this so long as they are strategically utilised.

Notable examples of this are expected to include the connection between the Navigation System and the Command System needing constant transmission from clocked navigation devices, and the Command System needing guaranteed maximum latency between itself and the ECS/FCS.

\begin{figure}[H]
\centering
\includegraphics[width=0.8\textwidth]{tsn_example}
\caption{VLAN and Priority Groupings}
\label{fig:vlans}
\end{figure}

\begin{figure}[H]
\centering
\includegraphics[width=0.6\textwidth]{tsn_routing}
\caption{Critical VLAN Topology}
\label{fig:topology}
\end{figure}

% TODO: table of time analysis of maximum hop time between some sample systems (prioritise DAL A/B)

\section{Fault Tree Analysis}
% 5 pages

\subsection{Example Fault Tree Analysis}
% Part A: ~1 page

% feedback: needs to talk about the answer to 1C.

\begin{figure}[H]
\centering
\includegraphics[width=0.85\textwidth]{tsn_fta}
\caption{Fault Tree Analysis~\cite{w.e.veselyFaultTreeHandbook1981} with Communication faults highlighted in blue, dotted lines are intended to reduce duplication}
\end{figure}

\subsection{Explanation}
% Part B: ~2 pages

In this FTA, we make the following assumptions and assertions:

\begin{itemize}
    \item For simplicity, the fuel pumping system is simplified to a pump, a valve and selectable (absolute) fuel tank input.
    \item For simplicity, many different points in the system where mechanical failure can manifest have been ignored, and mechanical failures which are illustrated have been expressed in a simplistic way.
    \item The engines are managed entirely by the ECS, and cannot be made to run dangerously by pilot error.
\end{itemize}

If an unwanted engine shutdown is to be caused (in whole or part) by the communication system, both of the following must occur:
\begin{enumerate}
    \item A fault in the primary ethernet system, either: \begin{enumerate}
        \item Messages are not delivered on time or at all
        \item Messages are delivered but are corrupt or invalid
    \end{enumerate}
    \item A fault in the backup ethernet system (\keyword{ALL-RED}) or a fault that prevents it from being used
\end{enumerate}
If these two faults occur simultaneously, the communications system can be solely responsible for the engine failure.
The one exception to this is in the event of a fuel flow failure: there only needs to be one timing fault (with PTP-GM) in the communications system to produce the hazard.

This idenfitifes timing errors in particular as being a critical problem and needing of particular attention to establish acceptable safety guarantees.
Timing faults may be produced by either clock drift in ethernet components (switches, interfaces or otherwise) which in turn is caused by incorrect communication between the PTP grandmaster and the component, or transmission starvation of high-criticality packets.

Transmission starvation probability is increased (in a non-scheduled system) as traffic volume increases.
The infotainment system in particular can be expected to produce a large amount of ethernet traffic to distribute high-quality audiovisual data across the aircraft, contributing to this effect.

\subsection{Mitigating Factors of Time Sensitive Networks}
% Part C: ~2 pages

Time Sensitive Networking provides several guarantees addressing the above phenomena:

\begin{itemize}
    \item Precision Time Protocol (PTP) as implemented for TSN by IEEE 802.1AS~\cite{IEEEStandardLocal2020} requires compliant devices to synchronise effectively under heavy load; additionally, the protocol implements an algorithm for selecting a new PTP grandmaster in the event of a failure of the current one.
    Thus, if there is a failure of both PTP-GM and any backup machines connected, one of the other stations can be selected as a reference leader in the meantime; an effective partial recovery from the fault which should prevent engine loss in the event of PTP-GM faults.
    \item Time-aware scheduler 802.1Qbv~\cite{IEEEStandardLocal2016a} functions by allocating time segments for transmission of different priority traffic, preventing transmission starvation.
    Operational time is broken into fixed-length cycles which have fixed-length slots within them for high-criticality traffic, during which no other traffic may be transmitted.
    If the time division scheme is configured appropriately, guaranteed maximum-delivery-time can be adjusted to provide appropriate safety guarantees and effectively mitigate the effect of high traffic on the engine hazard.
\end{itemize}

\section{Question 3}
% ~3.5 pages

\subsection{Safety Argument}
% Part A: ~1 page

\begin{figure}[h]
\centering
\includegraphics[width=0.85\textwidth]{tsn_gsn}
\caption{Argument presented with Goal-Structuring-Notation}
\end{figure}

\subsection{Explanation}
% Part B: ~1 page

This argument is divided into three broad sub-arguments relating to systems on the aircraft: \begin{description}
    \item{\textbf{Software}} the software driving the ECS units
    \item{\textbf{Hardware}} the mechanical systems on the engines
    \item{\textbf{Communications}} the physical equipment and firmware comprising the TSN-compliant ethernet system
\end{description}

\subsection{Evidence Gathering Techniques}
% Part C: ~1 page

\subsubsection*{Software}

% - software testing
%   - need to enumerate the expected conditions the system must function under, including partial failure and unusual real-world scenarios

To collect simulation data on ECS software, the final ECS hardware configuration (including the ECS units and command system) should be either assembled and outfitted with data collection equipment, or appropriately simulated in a time-aware fashion.
Ideally both approaches would be utilised.
System operation should be simulated with appropriate induced response from interfacing components --- for the purposes of testing the ECS software, this is primarily the Command System.
During simulated operation, data should collected on the operation of software on the ECS units.
The conditions of the system should be simulated across all conceivable performance parameters and the resulting data scrutinised for appropriate reaction and nominal control of engine function.

This highlights a difficulty of gathering comprehensive test in this fashion: correctly enumerating all possible expectable conditions and states.
Plotting the state space of conditions the aircraft may face is a highly involved task and must be informed by flight data from previous platforms and endeavours, in addition to periphery data such as atmospheric measurements taken at all times of year the aircraft is intended to fly.
Possible states of the engine hardware must also be known in advance, which must draw from design information of the engines and their own physical testing.
Once a comprehensive model of possible conditions is assembled, a simulation plan must be constructed which samples the space enough to give reasonable guarantees while remaining feasible to completely run. 

Formal proof of software function is complicated to derive, and depends upon the chosen software architecture and its available modelling methods.
It is assumed that an appropriate toolchain such as ADA SPARK~\cite{SPARK2014}, is used for this purpose and the relevant methods applied.
Independent code review should be performed by a mixed team of experts who are not otherwise involved with development of the platform.
Software Bill of Materials and design documentation is assumed to be produced during the development process itself.

\subsubsection*{Hardware}

To collect simulation data on the Mean-time-between-failures of engine hardware, final engines should be assembled and affixed to test rigs in controlled environments.
All conceivable and representable conditions should be simulated to the best of the ability of the rig and environment, and detailed data gathered on the engine's reaction to it; such data may include fuel consumption, spool RPM, thrust and deviations from expected responses.
Once more, the same issue of simulating all real-world conditions arises, and is further constrained by the limits of laboratory conditions.

At a later stage of development, the aircraft should be assembled and test flights conducted with similar data gathering methods in ``real'' conditions.

\subsubsection*{Communications System}

% - ethernet testing
%   - set up test rig, both in lab and on final aircraft
%   - test full operation, with simulated returns from ECS etc.
%   - vary conditions (eg. induce failure in components, add RF noise characteristic of smartphones, electrical storms, radar etc)
%   - measure both integrity and timing data

To test the custom softare implementing INT-FD, the same methods as with the ECS sofware should be used, but performed on the ethernet interfaces, with simulated failures in transmission on the primary ethernet system.

Integration testing should be performed in both lab conditions and on the final network as laid out in the airframe.
Full simulated conditions should be run in a similar method on these setups to the above described approaches with ECS software, but with special focus on timing guarantees and data transfer integrity.
Additional conditions that must be simulated for:\begin{itemize}
    \item Induced failure in components to force ALL-RED to cover faults in the primary system under conditions of complete or partial primary failure
    \item High-intensity RF background noise, of multiple profiles which effectively simulate modern handheld device emissions, atmospheric electrical storms, high-intensity ground-based radar, solar flares and other infrequent but feasible EM conditions.
    ``Realistic'' simulation of this is likely to present a challenge, as such interference may affect data gathering systems in the setup and requires highly specialist equipment for configurable transmission.
\end{itemize}

Network architectures should be formalised so that accurate models can be derived for responsiveness and network suitability can be proven.
